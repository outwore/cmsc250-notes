\section{Thursday, September 14, 2023}
\subsection{More on Predicate Logic: Truth and Falsity}
For universal and existential quantifiers, it is imperative that we establish how to find if an existential and/or universal statement is true or false.

\begin{itemize}
    \item To show $\exists$ statement is true, find an example in the domain where it is true.
    \item To show $\exists$ statement is false, show false for every member of the domain.
    \item To show $\forall$ statement is true, show true for every member of the domain.
    \item To show $\forall$ statement is false, find an example in the domain where it is false.
\end{itemize}

Do note that the domain does matter!

\begin{example}
\textbf{Is the following true:}\\
$\forall x \exists y: y<x$
\end{example}

To determine whether or not this is true, we must establish:
\begin{itemize}
    \item Is the domain $\mathbb{N}$? (Naturals)
    \item Is the domain $\mathbb{Z}$? (Integers)
    \item Is the domain $\mathbb{Q}$? (Rationals)
    \item Is the domain $\mathbb{Q}^>^0$? (Rationals that are greater than 0)
    \item Is the domain $\mathbb{Q}^>^=^0$? (Rationals that are greater than or equal to 0)
    \item Is the domain $\mathbb{R}$? (Reals)
    \item Is the domain $\mathbb{C}$? (Complex)
\end{itemize}

\subsection{Vacuity in Universal Statements}
\textbf{\underline{Definition 6.1}}: If domain, D, is empty, then $(\exists x \in D)[P(x)]$ is vacuously false. This then means that if domain, D, is empty, then $(\forall x \in D)[P(x)]$ is vacuously true.

\begin{example}
    All balls in the bowl are blue. \textbf{Is it true or false?}
\end{example}

This statement is false, if and only if, its negative is true. Its negation is "there exists a ball in the bowl that is not blue." But what if the bowl is empty?

If the bowl is empty, the negation would then be false as there does not exist a ball in the bowl that is not blue, which makes the original statement true by "default" or in other words, vacuously true.

\subsection{Order Matters}
The order for which domain values variables are placed in a predicate statement do indeed matter.

\begin{example}
    $Q(x,y,z): x+y=z$, for $x,y,z \in \mathbb{R}$\\
    \textbf{Are these statements equivalent?}\\
    $\forall x \forall y \exists z: Q(x,y,z)$\\
    $\exists z \forall x \forall y: Q(x,y,z)$
\end{example}

\textbf{Try to do this one yourself :)}

\subsection{Negating Nested Quantifiers}
Negation of quantifiers works exactly like how one would expect it to work, just like De Morgan's laws.

\begin{example}
    \textbf{Express the negation of the statement: }$\forall x \exists y: (xy=1)$ 
\end{example}

As typical, apply De Morgan's laws carefully and then you will get the final answer.

\begin{center}
    $\exists x \neg \exists y: (xy=1)$\\
    $\exists x \forall y: \neg (xy=1)$\\
    $\exists x \forall y: (xy\neq 1)$
\end{center}