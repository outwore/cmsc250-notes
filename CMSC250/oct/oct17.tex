\section{Tuesday, October 17, 2023}

\subsection{Proof by Induction}

To give ourselves a general basis on induction, we can give ourselves an example to alleviate the trouble of figuring it out for the first time.

Let $P(n)$ be the sentence "n cents postage can be obtained using 3 cent and 5 cent stamps."

\textbf{Main Idea}: We want to show that "$P(k)$ is true" implies that $P(k+1)$ is true, for all $k \geq 8$. An ideal solution here would be to setup 2 different cases:

\begin{itemize}
	\item Our first case is going to be showing that $P(k)$ is true, AND the k cents contain at least one 5 cent stamp.
	\item Our second case is going to be showing that $P(K)$ is true, AND the k cents do not containe ANY 5 cent stamps.
\end{itemize}

A basic idea regarding induction involves us thinking about how a chain of dominos falls. 

\begin{itemize}
	\item We know that the first domino will fall, since we are the ones knocking it over.
	\item Thus, we have to show that we can prove that every subsequent domino will fall over due to us knocking over the first domino.
\end{itemize}

From here, we can now set our basis on the definition of induction (in simple terms).

\subsubsection{Simple Induction}
\textbf{\underline{Definition 15.1}}: Let's claim that $\forall n \in \mathbb{N} : [P(n)]$.

\begin{proof}
	Proof by Induction\\
	By inducting on n, we may now form cases.\\
	\textbf{Base Case}: Show $P(0)$ directly.\\
	\textbf{Inductive Hypothesis}: Assume that $P(k)$ is true, for some $k \in \mathbb{N}$.\\
	\textbf{Inductive Step}: Prove that $P(k+1)$ must also be true based on the initial assumption that $P(k)$ is true.
\end{proof}

Let's take a look back at our example on dominos, and prove it through induction.

\begin{example}
	\textbf{Theorem}: $(\forall n \in \mathbb{N}^{>0})[$Domino $n$ will fall]
\end{example}

\begin{proof}
	Proof by Induction\\
	\textbf{Base Case}: The first domino [$P(0)$] will fall, because I knocked it over.\\
	\textbf{Inductive Hypothesis}: Assume domino $k$ will fall over, for some $k \in \mathbb{N}^{>0}$.\\
	\textbf{Inductive Step}: Since domino $k$ is falling, then domino $k+1$ will be struck by domino $k$, thus knocking it over (due to physics).
\end{proof}

Recall how we showed the Modular Arithmetic Theorem:\\

Let $a,b,c,d,n \in \mathbb{Z}$, where $n > 1$. Suppose $a \equiv_n c$ and $b \equiv_n d$. Then:

\begin{itemize}
	\item $a+c \equiv_n b+d$
	\item $ac \equiv_n bd$
	\item $a-c \equiv_n b-d$
	\item $a^m \equiv_n c^m$ for all $m \in \mathbb{N}$
\end{itemize}

\begin{example}
	\textbf{Theorem}: $(\forall n \in \mathbb{N}^{\geq 1})[n^3 \equiv_3 n]$
\end{example}

\begin{proof}
	Proof by Induction\\
	\textbf{Base Case}: Suppose $n=1$. Therefore, $n^3 \equiv_3 n$ turns into $1^3 \equiv_3 1$, which is true.
	\textbf{Inductive Hypothesis}: Assume k such that $k^3 \equiv_3 k$ is true, for some $k \in \mathbb{N}^{\geq 1}$.
	\textbf{Inductive Step}:
\end{proof}



